Let us now go about finding NP-complete problems. The first problem shown to be NP-complete is a problem in Boolean algebra known as the satisfiability problem, proven by Stephen Cook in 1971. Before we explore the proof, we give some terminology.

\subsection{Boolean algebra}

\begin{definition}
    A variable $u$ is a \textbf{Boolean variable} if $u \in \{\text{T},\text{F}\}$\footnote{Here T = True, F = False}. The \textbf{complement} of $u$, denoted $\overline{u}$, is defined by
    \begin{equation*}
        \overline{u} =
        \begin{cases}
            \text{F if $u=\text{T}$}\\
            \text{T if $u=\text{F}$}
        \end{cases}
    \end{equation*}
\end{definition}

\begin{definition}
    Let $U = \{u_1,u_2,\dots,u_m\}$ be a set of boolean variables. Then,

    \begin{itemize}
        \item A \textbf{clause} over $U$ is a set created by choosing at most one element from the sets \newline $\{u_1,\overline{u_1}\},\{u_2,\overline{u_2}\},\dots\{u_m,\overline{u_m}\}$,

        \item A \textbf{truth assignment} is a function $t:U \to \{T,F\}$,

        \item A clause is \textbf{satisfied} by a truth assignment if, under the truth assignment given, at least one member of the clause is True. A collection $C$ of clauses over $U$ (in Conjunctive Normal Form) is \textbf{satisfiable} if there exists some truth assignment such that each element of $C$ is satisfied under that truth assignment.
    \end{itemize}
\end{definition}

\begin{example}
    Consider the set of boolean variables $\{u_1,u_2\}$. The clause $\{u_1,\overline{u_2}\}$ can be satisfied by any truth assignment such that \textit{either} of $t(u_1)=T$, $t(u_2) = F$ hold.\\

    Now consider the set of clauses $\{\{u_1,\overline{u_2}\},\{\overline{u_1},u_2\}\}$. This is satisfiable and can be satisfied by $t(u_1) = t(u_2) = T$. On the other hand, the collection $\{\{u_1,\overline{u_2}\},\{u_1,u_2\},\{\overline{u_2}\}\}$ cannot be satisfied by any truth assignment and so is unsatisfiable.
\end{example}

\subsection{The satisfiability problem}
We now turn to the satisfiablity decision problem, and Cook's seminal proof.

\decisionproblem{Satisfiability (Sat)}{A set $U$ of boolean variables and a collection $C$ of clauses over $U$}{Is there a satisfying truth assignment for $C$?}

%\medskip

The theorem is now given.

\begin{theorem}
    (\cite{CookStephenA.1971CoPM}) \textsc{Satisfiability} is NP-complete.
\end{theorem}

\begin{proof}
    We give a sketch proof based on (\cite{GareyMichaelR1979Cai:}). First we acknowledge that $\text{\textsc{Sat}} \in \mathcal{NP}$, as given a truth assignment that it satisfies $C$ in polynomial time. Thus our first requirement is met.

    For the second requirement, we consider the set of languages $L_{\textsc{Sat}}$ which are represented by \textsc{Sat}. We need to show that for all languages $L \in \mathcal{NP}$, $L \propto L_{\text{\textsc{Sat}}}$. As there are an infinite number of languages in $\mathcal{NP}$, we do not create a transition function between the languages. Instead we inspect a polynomial time program in a NDTM that recognises them, as these will have certain distinguishing features. This gives a proof that acts simultaneously for all languages $L \in \mathcal{NP}$.

    Consider a polynomial time NDTM program $M = (\Gamma,Q,\delta)$ with alphabet $\Sigma$ which recognises an arbitrary language $L$ in polynomial time. That is, for any computation of $x \in L$ with $|x| = n$, there exists some polynomial $p$ such that $T_M(n) < p(n)$. Denote the polynomial transformation we construct by $f_L$, and instead of creating a polynomial transformation from $L$ to $L_{\text{\textsc{Sat}}}$ we create a mapping from $\Gamma$ to instances of \textsc{Sat}. This has the property that for all instances $x \in \Sigma^*$, $x \in L$ if and only if $f_L(x)$ has a satisfying truth assignment.

    If an instance $x \in \Sigma^*$ is accepted by $M$, then there must exist an accepting computation such that the number of steps in the checking stage are bounded by $p(n)$. In particular, we must only utilise the tape squares $-p(n)$ to $(p(n)$. Hence, we can describe the NDTM at time $0 \leq t \leq p(n)$ by:

    \begin{itemize}
        \item The contents of the tape squares;
        \item The state of $M$;
        \item The position of the read-write head.
    \end{itemize}

    Since there are a finite number of these, we can assign them to be boolean variables, for which we call the variable set $U$. Label $Q = \{q_0,q_1= q_Y, q_2 = q_N, q_3, \cdots , q_r\}$ where $r = |Q| -1$, and label $\Gamma = \{s_0 = b,s_1,\cdots,s_v\}$ where $v = |\Gamma| - 1$. Below is a table of the description of the variables.

    \begin{table}[h]
    \begin{center}
    \begin{tabular}{|c|c|c|}\hline
        Variable & Range & Meaning \\ \hline
        \multirow{2}{5em}{$Q[i,k]$} & $0 \leq t \leq p(n)$ & \multirow{2}{20em}{At time $t$, $M$ is in stage $q_k$.}\\
        & $0 \leq k \leq r$ & \\ \hline
        
        \multirow{2}{5em}{$H[i,j]$} & $0 \leq t \leq p(n)$ & \multirow{2}{20em}{At time $t$, the read-write head is scanning tape square $j$.} \\
        & -p(n) \leq j \leq p(n)+1 & \\ \hline

        \multirow{3}{5em}{$S[i,j,k]$} & $0 \leq  t \leq p(n)$ & \multirow{3}{20em}{At time $t$, the contents of tape square $j$ is symbol $s_k$.} \\
        & $-p(n) \leq j \leq p(n)+1$ & \\
        & $0 \leq k \leq v$ & \\ \hline
    \end{tabular}
    \end{center}
    \caption{A table of the variables and their meanings.}
    \label{table:Variables}
\end{table}


    
    When $M$ computes $x$, it induces a truth assignment on the variables. The construction of $f_L$ is such that an accepting computation on $x$ in polynomial time will return a truth assignment that satisfies $U$. Given this construction, we then have
    \begin{align*}
        x \in L &\iff \text{there is an accepting computation of $M$ on $x$}\\
        &\iff \text{there is an accepting computation of $M$ on $x$ that takes $p(n)$ or fewer steps}\\
        &\iff \text{there is a satisfying truth assignment for the collection of clauses in $f_L(x)$.}
    \end{align*}

    This means that $f_L$ satisfies one of the two requirements required of a polynomial transformation. We now complete our description of $f_L$.

    \begin{table}[h]
    \begin{center}
    \begin{tabular}{|c|c|}\hline
       Clause group & Restriction imposed \\ \hline
        $G_1$ & \multirow{1}{20em}{At time $t$, $M$ is in exactly 1 state} \\
         
        $G_2$ & \multirow{2}{20em}{At time $t$, the read-write head is scanning exactly one tape square}\\ &  \\
        
        $G_3$ & \multirow{2}{20em}{At time $t$, each tape square contains exactly one symbol from $\Gamma$}\\ &  \\
        
        $G_4$ & \multirow{2}{20em}{At time $0$, the computation is in the initial configuration of its checking stage for input $x$}\\ & \\
        
        $G_5$ & \multirow{2}{20em}{By time $p(n)$, $M$ has entered state $q_Y$ and hence has accepted $x$}\\ & \\
        
        $G_6$ & \multirow{3}{20em}{For each time $0 \leq t \leq p(n)$, the configuration of $M$ at time $t+1$ follows a single application of the transition function of $M$ at time $t$.}\\ & \\ & \\ \hline
    \end{tabular}        
    \end{center}
    \caption{A table of clause groups and their meanings.}
    \label{figClauseGroups}
\end{table}

    The table of clause groups clearly specifies a truth assignment such that if $x \in L$, then this truth assignment is satisfied. However, if $x \notin L$, then either it must be rejecting or does not halt (in which case clauses $G_5$ is violated) which does not satisfy the truth assignment. Hence $x \in L$ if and only if there is some satisfying truth assignment for the clause group given above.

    What is now left to show is the for a fixed language $L$, $f_L(x)$ can be constructed in time bounded by some polynomial function of $n = |x|$. But the choice of our clauses was arbitrary, so we can adjust the clause group for our satisfying computation.
 
    Our final task is to show that the time taken to encode \textsc{Sat} will be bounded above by some polynomial. This is easily done; the collection of clauses is polynomial on $n$, as is the number of possible sets of clauses, so the size of encoding is also polynomial on $n$ as desired.

    Hence, for any language $L \in \mathcal{NP}$, $f_L$ acts as a polynomial transformation from $L$ to $L_{\textsc{Sat}}$. It follows that \textsc{Sat} is NP-complete.
\end{proof}