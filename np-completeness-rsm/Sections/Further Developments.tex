In this report we have introduced and formalised the Turing Machine and demonstrated its computational effectiveness compared to other mechanical methods of operation. Through this, the Turing Machine has been shown to compute languages with different complexity classes, of which we have layed out a foundation to prove whether or not a language is in $\mathcal{P}$ or $\mathcal{NP}$.

Following the proof of Cook's Theorem in 1971, proving the existence of NP-complete problems became far easier, as if one can prove there exists a polynomial transformation from \textsc{Sat} to some decision problem $\Pi$, then $\Pi$ is NP-complete by the transitivity of polynomial transformations. Richard Karp implemented this technique to show that there exist far more problems that are NP-complete (\cite{Karp1972}). These include the vertex cover problem (determining the minimum number of vertices that are at the endpoint of every edge) and the knapsack problem (determining, given a set of items with weights and value, the maximum value possible while keeping the total weight under a certain amount).

While a large focus of complexity theory is on finding methods to show that $\mathcal{P} = \mathcal{NP}$, there have also been methods discovered that could be used towards a proof of inequality. The only known technique that can be used for this is a technique known as \textbf{diagonalisation} (\cite{AroraSanjeev2009Cc:a}). While a proof or disproof of the famed conjecture is yet to be found, this technique was used to prove that if $\mathcal{P} \neq \mathcal{NP}$, there exist languages that are neither in $\mathcal{P}$ nor NP-complete (\cite{LadnerRichardE.1975OtSo}).