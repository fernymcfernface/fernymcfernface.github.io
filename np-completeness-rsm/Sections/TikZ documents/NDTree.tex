\begin{figure}[h]
    \centering
    \parbox{5in}{
    \begin{tikzpicture}[thick, empty/.style={color = white, text=black}]
        \graph [tree layout, nodes = {circle, draw, minimum size=2em}, level distance = 3em] {
        "" -> 
        {"" -> 
        {"" -> qN1/"$q_N$"
            , "" ->
            {"" ->
            {"" -> {"" -> qN2/"$q_N$", "" -> H1/"$\vdots$" [empty]}},"" -> {"" -> "" -> H2/"$\vdots$" [empty],"" -> "" -> H3/"$\vdots$" [empty]}
        }
        }
        , "" -> {"" -> {qN3/"$q_N$",qN4/"$q_N$"},"" -> qN5/"$q_N$",{"" -> {"" -> {"" -> {"" -> "$q_Y$"},"" -> "" -> H4/"$\vdots$" [empty]}, "" -> qN6/ "$q_N$"}}}}
        };
    \end{tikzpicture}}
%---------------------------------
    \parbox{1in}{
    \begin{tikzpicture}[thick]
        \graph [grow down=3em, nodes = {circle,draw, minimum size = 2em}, level distance = 3em] {"" -> "" -> "" -> "" -> "" -> "" -> "$q_Y$"};
    \end{tikzpicture}}
    \caption{A graph of a nondeterministic computation alongside a deterministic computation. A branch of the computation that does not halt is indicated by $\vdots$. For all branches, only the halting state is shown.} 
    \label{fig:NDTree}
\end{figure}