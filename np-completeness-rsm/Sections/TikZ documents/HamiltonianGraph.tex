\begin{figure}
    \centering
    \begin{tikzpicture}[thick, main/.style = {draw,circle}]
        \node (1) [circle, draw] at (0,0) {1};
        \node (2) [circle, draw] at (3,0) {2};
        \node (3) [circle, draw] at (6,0) {3};
        \node (4) [circle, draw] at (6,-3) {4};
        \node (5) [circle, draw] at (3,-3) {5};
        \node (6) [circle, draw] at (0,-3) {6};
    
        \graph{ (1) --  (2) -- (3) -- (4) -- (5) -- (6) -- (1) };
        \graph{ (4) -- (2) -- (6) };
        
    \end{tikzpicture}
    \hfill
    \begin{tikzpicture}[thick, main/.style = {draw,circle}]
        \node (1) [circle, draw] at (0,0) {1};
        \node (2) [circle, draw] at (3,0) {2};
        \node (3) [circle, draw] at (6,0) {3};
        \node (4) [circle, draw] at (6,-3) {4};
        \node (5) [circle, draw] at (0,-3) {5};
    
        \graph{ (5) -- (1) --  (2) -- (3) -- (4) };
        \graph{ (4) -- (2) -- (5) };
        
    \end{tikzpicture}
    \caption{Two graphs. The first contains a Hamiltonian circuit (via the path $(1,2,3,4,5,6,1)$) while the second does not.}
    \label{fig:HamiltonianGraph}
\end{figure}